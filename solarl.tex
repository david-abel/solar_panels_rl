\documentclass[11pt]{article}

% --- Packages ---
\usepackage[usenames, dvipsnames]{color} % Cool colors
\usepackage{enumerate, amsmath, amsthm, amssymb, fullpage, csquotes, dashrule, tikz, bbm, booktabs, bm}
\usepackage[framemethod=TikZ]{mdframed}
\usepackage[numbers]{natbib}

% --- Misc. ---
\hbadness=10000 % No "underfull hbox" messages.
\setlength{\parindent}{0pt} % Removes all indentation.

% --- Commands ---
\input{commands}

% --- Fancy Headers ---
\usepackage{fancyhdr}
\fancypagestyle{dave}
{
	\fancyhf{} % sets both header and footer to nothing
	\renewcommand{\headrulewidth}{0pt}
	\cfoot{Page \thepage}
	\rfoot{Emily Reif \& David Abel}
	\lfoot{Solar Panels and RL}
}
\pagestyle{dave}

% --- Meta Info ---
\title{Solar Panels and RL}
\author{Emily Reif \& David Abel}
\date{October 19, 2016}

% --- Begin Document ---
\begin{document}
\maketitle
% Footer
\thispagestyle{dave}

\section{Introduction}
The use of solar panels offers a pollution free, sustainable means of harvesting energy directly from the sun. Much work has investigating how to maximize the efficiency of the end to end system, including the design of photovolatic cells, the layout of the panels, and tracking systems. Solar panels equipped with a tracking mechanism are designed with either one or two degrees of freedom; the tracker follows the sun throughout the day, and the panel points directly at the sun to minimize the angle of incidence between incoming solar irradiance and photovoltatic cells, as in ~\citet{Eke2012,Benghanem2011,King2001}. In this work, we advance the computational paradigm of Reinforcement Learning (RL) to optimize solar panel performance, measured in the total amount of energy harvested in a 24 hour cycle. We advocate for the use of RL due to its {\it effectiveness}, {\it negligible cost}, {\it lack of dependence on extra components} such as a GPS, and {\it versatility}. We justify each of these properties, develop a new RL algorithms, \textsc{SOLARL} designed specifically for harvesting solar energy, create a simulation platform for solar energy harvesting, and test the utility of our algorithm in this simulated environment and on a fleet of real solar panels.



\section{Evaluation}

The metric we care about is {\it average energy harvested in a day}. How might we test this in simulation?
\begin{itemize}
\item Video of the sky throughout the day
\item Use the base algorithm as a guide?
\item Get a single solar panel, collect data about how its tilted.
\item Is this really a {\bf contextual bandit problem?} This is absolutely worth thinking about.
\end{itemize}



So assuming we have a video stream of the sky. What then? We also need the {\it reward function} to come from somewhere. Really we have to figure out three things for the simulation:
\begin{enumerate}
\item What are the observations?
\item What is $\mc{R}(s,a)$?
\item What is $\mc{T}(s,a,s')$? This comes immediately from a selection for the action space (what things can the panel do?)
\end{enumerate}

A final question - how many panels should we be evaluating?

\subsection{Are there delayed rewards?}






\section{System Overview}













% --- Bibliography ---
\bibliographystyle{plainnat}
\bibliography{solar}

\end{document}